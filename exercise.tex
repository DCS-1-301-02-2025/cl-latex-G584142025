\documentclass[a4paper,11pt,dvipdfmx]{ujarticle}
% パッケージ
\usepackage{graphicx}
\usepackage{url}
% レイアウト指定を記述したファイルの読み込み
\input{layout}

% タイトルと氏名を変更せよ.
\title{日本におけるデジタル化の状況}
\author{G584142025 MUHAMMAD YASSIR RAMADHAN}

\begin{document}

\maketitle %ここにタイトルが入る

% ここから本文
% 節見出し: \section{}
% を使う
\section{ブロードバンドの整備状況}

% 本文(1)
%  参考文献の参照: \cite{}
%  図番号の参照: \ref{}
% を使う
% 文献データベースのキーワードは oecd と imd
% になっている.
OECDによるブロードバンド回線の
普及に関する調査\cite{oecd}によると、図\ref{fiber}に示すように,
日本における 100人あたりの光ファイバー回線の
加入者数は29.0で、韓国,スウェーデン、ノルウェー
に続いて第4位になっている.

% 図の挿入
% \includegraphics{}
% を
% \begin{figure}[htbp]
% \end{figure}
% で囲み
% \caption{}
% で図のタイトルを入れる.
% \label{}
% を使って図番号が参照できるようにする
% また,
% \centering
% で図が中央に来るようにする
\begin{figure}[htbp]
    \centering
    \includegraphics{latexpng.png}
    \caption{光ファイバー回線の加入者数(100人あたり)}
    \label{fiber}
\end{figure}

% 本文(1)
% ーーー
% 節見出し(2)
\section{デジタル競争力ランキング}

% 本文(2)
国際経営開発研究所(IMD)の調査\cite{imd}によると、
日本のデジタル競争力のランキングは表\ref{degirank}に示すように、
調査対象の64カ国中、総合で28位、
準備分野で30位となっている。

% 表の挿入
% \begin{tabular}
% \end{tabular}    
% による表の記述を 
% \begin{table}[htbp]
% \end{table}
% で囲み
% \caption{}
% で表のタイトルを入れる.
% \label{}
% を使って表番号が参照できるようにする
% また,
% \centering
% で表が中央に来るようにする
\begin{table}[htbp]
    \centering
    \caption{デジタル競争力ランキング(64カ国中)}
    \label{degirank}
    \begin{tabular}{|c|c|c|}
        \hline
         国 & 総合 & 準備 \\
         \hline
         米国 & 1位 & 4位 \\
         \hline
         香港 & 2位 & 10位 \\
         \hline
         スウェーデン & 3位 & 8位 \\
         \hline
         デンマーク & 4位 & 2位 \\
         \hline
         シンガポール & 5位 & 3位 \\
        \hline
        \hline
         韓国 & 12位 & 13位 \\
         \hline
         中国 & 15位 & 20位 \\
         \hline
         \hline
         日本 & 28位 & 30位 \\
         \hline
     \end{tabular}
\end{table}

% ーーー
% 見出し(3)
\section{考察}
% 考察
%
% \begin{itemize}
% \end{itemize}
% を使って箇条書きで記述する
\begin{itemize}
    \item 日本のブロードバンド整備
    \begin{itemize}
        \item 日本は世界第4位のブロードバンド普及率を誇り、全国的に高速で安定したインターネット環境が整備されている。このようなインフラの充実は、情報への迅速なアクセスやオンラインサービスの活用を可能にし、国民の生活の質を大きく向上させている。特に、リモートワークやオンライン教育の普及においては、こうした整備状況が重要な役割を果たしていると言える。今後も安定性やセキュリティを含めたインフラのさらなる高度化が求められるだろう。
    \end{itemize}
\end{itemize}
\begin{itemize}
    \item 日本のデジタル競争力
    \begin{itemize}
        \item 一方で、日本のデジタル競争力は64カ国中28位にとどまっており、インフラの整備状況と比較すると低い順位にある。このギャップは、単にインターネット環境が整っているだけでは、国際的な競争力の向上には直結しないことを示している。デジタル人材の不足、企業や行政のデジタル化の遅れ、教育現場でのIT活用の遅れなど、ソフト面での課題が依然として多い。今後は、インフラの利点を活かすためにも、教育や制度改革を通じて、国全体のデジタル対応力を底上げしていく必要があると思います。
        
         
    \end{itemize}
\end{itemize}

% ここに参考文献が入る
%
\bibliographystyle{junsrt}
\bibliography{exercise.bib}

\end{document}